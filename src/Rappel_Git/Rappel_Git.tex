%& --output-directory=../../pdf/

\documentclass{beamer}
\usepackage[frenchb]{babel}
\usepackage[T1]{fontenc}
\usepackage[utf8]{inputenc}
\usetheme{CambridgeUS}
\usepackage{tikz}
\usepackage{listings}
\usepackage{verbatim}
\usepackage{hyperref}

\usetikzlibrary{arrows,shapes}
\lstset { %
	language=bash,
	backgroundcolor=\color{black!5}, % set backgroundcolor
	basicstyle=\footnotesize,% basic font setting
}

\title{IN104 - Solveur de Sudoku}
\author{Ugo Vollhardt}
\institute{CEA LIST}

\begin{document}
	\pgfdeclarelayer{background}
	\pgfsetlayers{background,main}
	\section{Rappels Git}
	\begin{frame}
	\frametitle{L'espace de travail}
	\begin{center}
	\begin{tikzpicture}
		\coordinate (lu1) at (0,3);
		\coordinate (lu2) at (4,3);
		\coordinate (lu3) at (8,3);
		\coordinate (rb1) at (3,0);
		\coordinate (rb2) at (7,0);
		\coordinate (rb3) at (11,0);
		\fill[color=gray!20] (lu1) rectangle (rb1);
		\draw (lu1) rectangle (rb1);
		\draw (lu1) node[below right]{\footnotesize Depôt local};
		\fill[color=gray!20] (lu2) rectangle (rb2);
		\draw (lu2) rectangle (rb2);
		\draw (lu2) node[below right]{\footnotesize Staging area};
		\fill[color=gray!20] (lu3) rectangle (rb3);
		\draw (lu3) rectangle (rb3);
		\draw (lu3) node[below right]{\footnotesize Répertoire de travail};
		\draw[dashed] (7.5,-2) -- (7.5,5);
		\draw[->,>=latex] (9,0) to[bend left] (6,0);
		\draw (3.5,-0.5) node[below]{\footnotesize git commit};
		\draw[->,>=latex] (5,0) to[bend left] (2,0);
		\draw (7.5,-0.5) node[below]{\footnotesize git add};
		\draw[->,>=latex] (2,3) to[bend left] (9,3);
		\draw (5.5,4) node[below]{\footnotesize git checkout};
		\draw[->,>=latex] (6,3) to[bend left] (9,3);
		\draw (7.5,3.5) node[below]{\footnotesize git reset};
		\draw (7.5,5) node[below right]{Visible par l'utilisateur};
		\draw (7.5,5) node[below left]{Caché par Git};
	\end{tikzpicture}	
	\end{center}
	\end{frame}


\begin{frame}
	\frametitle{Les Commits}
	Contenu d'un commit : 
	\begin{itemize}
		\item Un nom
		\item Une description
		\item Une liste de modifications
		\item Un parent
	\end{itemize}
	Création d'un Commit :
	\begin{itemize}
		\item {\bfseries \$ git status }: Affiche le status actuel du répertoire de travail (fichiers modifiés, ajoutés ou supprimés)
		\item {\bfseries \$ git add <file> }: Ajoute les modifications faites sur un fichier à la staging area. \\
		\item {\bfseries \$ git commit -m "Description"} : Forme un commit qui contiendra l'ensemble des modification présentes dans la staging area.
	\end{itemize}
\end{frame}
\begin{frame}[fragile]
	\frametitle{Le fichier .gitignore}
	\begin{columns}[c]
		\begin{column}{5cm}
			Pour un projet utilisant le langage C : \\
			\begin{verbatim}
			# fichiers de compilation
			*.o
			# Fichiers compilés
			*.lib
			*.a
			*.exe
			\end{verbatim}
		\end{column}
		\begin{column}{5cm}
			Permet de spécifier à Git d'ignorer complétement ces fichiers et les modifications faites dessus. \\
			$\Rightarrow$ Permet de ne gérer que les fichiers utiles
		\end{column}
	\end{columns}
\end{frame}

\begin{frame}
\frametitle{Contenu du depôt local : Historique des commits}
\begin{center}
	\begin{columns}[c]
		\begin{column}{5cm}
			\begin{center}
				\begin{tikzpicture}
				\node[draw,circle,fill=gray!20] (c1) at (0,0) {c1};
				\node[draw,circle,fill=gray!20] (c2) at (2,1.5) {c2};
				\node[draw,circle,fill=gray!20] (c3) at (0,1.5) {c3};
				\node[draw,circle,fill=gray!20] (c4) at (0,3) {c4};
				\node[draw,circle,fill=gray!20] (c5) at (0,4.5) {c5};
				\draw[->,>=latex] (c1) to[out=90, in=-90] (c2);
				\draw[->,>=latex] (c1) to[out=90, in=-90] (c3);
				\draw[->,>=latex] (c3) to[out=90, in=-90] (c4);
				\draw[->,>=latex] (c4) to[out=90, in=-90] (c5);
				\node[draw,rectangle,fill=gray!50] (Branch) at (2,3) {Branch\_1};
				\node[draw,rectangle,fill=gray!50] (HEAD) at (1.5,5.5) {HEAD};
				\node[draw,rectangle,fill=gray!50] (Master) at (1.5,4.5) {Master};
				\draw[->,>=latex] (HEAD) to[out=-90, in=90] (Master);
				\draw[->,>=latex] (Master) to[out=180, in=0] (c5);
				\draw[->,>=latex] (Branch) to[out=-90, in=90] (c2);
				\end{tikzpicture}
			\end{center}
		\end{column}
		\begin{column}{5cm}
		\end{column}
	\end{columns}
\end{center}
\end{frame}

\begin{frame}
\frametitle{Contenu du depôt local : Historique des commits}
\begin{columns}[c]
	\begin{column}{5cm}
		\begin{center}
			\begin{tikzpicture}
			\node[draw,circle,fill=gray!20] (c1) at (0,0) {c1};
			\node[draw,circle,fill=gray!20] (c2) at (2,1.5) {c2};
			\node[draw,circle,fill=gray!20] (c3) at (0,1.5) {c3};
			\node[draw,circle,fill=gray!20] (c4) at (0,3) {c4};
			\node[draw,circle,fill=gray!20] (c5) at (0,4.5) {c5};
			\draw[->,>=latex] (c1) to[out=90, in=-90] (c2);
			\draw[->,>=latex] (c1) to[out=90, in=-90] (c3);
			\draw[->,>=latex] (c3) to[out=90, in=-90] (c4);
			\draw[->,>=latex] (c4) to[out=90, in=-90] (c5);
			\node[draw,rectangle,fill=gray!50] (Branch) at (2,3) {Branch\_1};
			\node[draw,rectangle,fill=gray!50] (HEAD) at (-1.5,1.5) {HEAD};
			\node[draw,rectangle,fill=gray!50] (Master) at (1.5,4.5) {Master};
			\draw[->,>=latex] (HEAD) to[out=0, in=180] (c3);
			\draw[->,>=latex] (Master) to[out=180, in=0] (c5);
			\draw[->,>=latex] (Branch) to[out=-90, in=90] (c2);
			\end{tikzpicture}
		\end{center}
	\end{column}
	\begin{column}{5cm}
		\$ git checkout c3
	\end{column}
\end{columns}
\end{frame}

\begin{frame}
\frametitle{Contenu du depôt local : Historique des commits}
\begin{columns}[c]
	\begin{column}{5cm}
		\begin{center}
			\begin{tikzpicture}
			\node[draw,circle,fill=gray!20] (c1) at (0,0) {c1};
			\node[draw,circle,fill=gray!20] (c2) at (2,1.5) {c2};
			\node[draw,circle,fill=gray!20] (c3) at (0,1.5) {c3};
			\node[draw,circle,fill=gray!20] (c4) at (0,3) {c4};
			\node[draw,circle,fill=gray!20] (c5) at (0,4.5) {c5};
			\draw[->,>=latex] (c1) to[out=90, in=-90] (c2);
			\draw[->,>=latex] (c1) to[out=90, in=-90] (c3);
			\draw[->,>=latex] (c3) to[out=90, in=-90] (c4);
			\draw[->,>=latex] (c4) to[out=90, in=-90] (c5);
			\node[draw,rectangle,fill=gray!50] (Branch) at (2,3) {Branch\_1};
			\node[draw,rectangle,fill=gray!50] (HEAD) at (4,3) {HEAD};
			\node[draw,rectangle,fill=gray!50] (Master) at (1.5,4.5) {Master};
			\draw[->,>=latex] (HEAD) to[out=180, in=0] (Branch);
			\draw[->,>=latex] (Master) to[out=180, in=0] (c5);
			\draw[->,>=latex] (Branch) to[out=-90, in=90] (c2);
			\end{tikzpicture}
		\end{center}
	\end{column}
	\begin{column}{5cm}
		\$ git checkout c3 \\
		\$ git checkout Branch\_1 \\
	\end{column}
\end{columns}
\end{frame}

\begin{frame}
\frametitle{Contenu du depôt local : Historique des commits}
\begin{columns}[c]
	\begin{column}{5cm}
		\begin{center}
			\begin{tikzpicture}
			\node[draw,circle,fill=gray!20] (c1) at (0,0) {c1};
			\node[draw,circle,fill=gray!20] (c2) at (2,1.5) {c2};
			\node[draw,circle,fill=gray!20] (c3) at (0,1.5) {c3};
			\node[draw,circle,fill=gray!20] (c4) at (0,3) {c4};
			\node[draw,circle,fill=gray!20] (c5) at (0,4.5) {c5};
			\draw[->,>=latex] (c1) to[out=90, in=-90] (c2);
			\draw[->,>=latex] (c1) to[out=90, in=-90] (c3);
			\draw[->,>=latex] (c3) to[out=90, in=-90] (c4);
			\draw[->,>=latex] (c4) to[out=90, in=-90] (c5);
			\node[draw,rectangle,fill=gray!50] (Branch) at (2,3) {Branch\_1};
			\node[draw,rectangle,fill=gray!50] (HEAD) at (1.5,5.5) {HEAD};
			\node[draw,rectangle,fill=gray!50] (Master) at (1.5,4.5) {Master};
			\draw[->,>=latex] (HEAD) to[out=-90, in=90] (Master);
			\draw[->,>=latex] (Master) to[out=180, in=0] (c5);
			\draw[->,>=latex] (Branch) to[out=-90, in=90] (c2);
			\end{tikzpicture}
		\end{center}
	\end{column}
	\begin{column}{5cm}
		\$ git checkout c3 \\
		\$ git checkout Branch\_1 \\
		\$ git checkout master \\
	\end{column}
\end{columns}
\end{frame}

\begin{frame}
\begin{columns}[c]
	\begin{column}{5cm}
		\begin{center}
			\begin{tikzpicture}
			\node[draw,circle,fill=gray!20] (c1) at (0,0) {c1};
			\node[draw,circle,fill=gray!20] (c2) at (2,1.5) {c2};
			\node[draw,circle,fill=gray!20] (c3) at (0,1.5) {c3};
			\node[draw,circle,fill=gray!20] (c4) at (0,3) {c4};
			\node[draw,circle,fill=gray!20] (c5) at (0,4.5) {c5};
			\draw[->,>=latex] (c1) to[out=90, in=-90] (c2);
			\draw[->,>=latex] (c1) to[out=90, in=-90] (c3);
			\draw[->,>=latex] (c3) to[out=90, in=-90] (c4);
			\draw[->,>=latex] (c4) to[out=90, in=-90] (c5);
			\node[draw,rectangle,fill=gray!50] (Branch) at (2,3) {Branch\_1};
			\node[draw,rectangle,fill=gray!50] (HEAD) at (-1.5,1.5) {HEAD};
			\node[draw,rectangle,fill=gray!50] (Master) at (1.5,4.5) {Master};
			\draw[->,>=latex] (HEAD) to[out=0, in=180] (c3);
			\draw[->,>=latex] (Master) to[out=180, in=0] (c5);
			\draw[->,>=latex] (Branch) to[out=-90, in=90] (c2);
			\end{tikzpicture}
		\end{center}
	\end{column}
	\begin{column}{5cm}
		\$ git checkout c3 \\
		\$ git checkout Branch\_1 \\
		\$ git checkout master \\
		\$ git checkout master$\sim$2 \\
	\end{column}
\end{columns}
\end{frame}

\begin{frame}
\frametitle{Lien avec un depôt distant}
\begin{center}
	\begin{tikzpicture}
	\coordinate (lu1) at (0,3);
	\coordinate (lu2) at (4,3);
	\coordinate (lu3) at (8,3);
	\coordinate (lu4) at (-5,3);
	\coordinate (rb1) at (3,0);
	\coordinate (rb2) at (7,0);
	\coordinate (rb3) at (11,0);
	\coordinate (rb4) at (-2,0);
	\fill[color=gray!20] (lu1) rectangle (rb1);
	\draw (lu1) rectangle (rb1);
	\draw (lu1) node[below right]{\footnotesize Depôt local};
	\fill[color=gray!20] (lu2) rectangle (rb2);
	\draw (lu2) rectangle (rb2);
	\draw (lu2) node[below right]{\footnotesize Staging area};
	\fill[color=gray!20] (lu3) rectangle (rb3);
	\draw (lu3) rectangle (rb3);
	\draw (lu3) node[below right]{\footnotesize Répertoire de travail};
	\fill[color=gray!20] (lu4) rectangle (rb4);
	\draw (lu4) rectangle (rb4);
	\draw (lu4) node[below right]{\footnotesize Depôt distant};
	\draw[dashed] (7.5,-2) -- (7.5,5);
	\draw[->,>=latex] (9,0) to[bend left] (6,0);
	\draw (3.5,-0.5) node[below]{\footnotesize git commit};
	\draw[->,>=latex] (5,0) to[bend left] (2,0);
	\draw (7.5,-0.5) node[below]{\footnotesize git add};
	\draw[->,>=latex] (2,3) to[bend left] (9,3);
	\draw (5.5,4) node[below]{\footnotesize git checkout};
	\draw[->,>=latex] (6,3) to[bend left] (9,3);
	\draw (7.5,3.5) node[below]{\footnotesize git reset};
	\draw[dashed] (-1,-2) -- (-1,5);
	\draw[->,>=latex] (1,0) to[bend left] (-3,0);
	\draw (-1,-0.5) node[below]{\footnotesize git push};
	\draw[->,>=latex] (-3,3) to[bend left] (1,3);
	\draw (-1,4) node[below]{\footnotesize git fetch};
	\draw (1.5,4.5) node[below]{\footnotesize git pull = git fetch + git checkout};
	\draw (-1,5) node[below right]{Ordinateur local};
	\draw (-1,5) node[below left]{Serveur distant};
	\end{tikzpicture}	
\end{center}
\end{frame}

\begin{frame}
\frametitle{Interaction avec un projet distant}
Initialisation : 
	\begin{enumerate}
		\item Clonage du projet sur un ordinateur local \\
			{\bfseries \$ git clone git@github.com$\colon$user/project.git}
	\end{enumerate}
Utilisation : 
	\begin{enumerate}
		\item Récupération du contenu du serveur distant \\
		{\bfseries \$ git pull}
		\item Publication du contenu local sur le serveur distant \\
		{\bfseries \$ git push}
	\end{enumerate}
\vskip 0.5cm
Remarque : Privilégier l'utilisation de l'adresse SSH plutôt que HTTPS
\end{frame}

\begin{frame}
\frametitle{Ressources utiles}
\begin{itemize}
	\item Fiche mémoire git : \\
	\url{https://github.com/UgoVollhardt/CheatSheetGit/blob/master/CheatSheet.pdf}
	\item Tutoriel interactif Git : \\
	\url{https://learngitbranching.js.org/}
	\url{https://try.github.io/levels/1/challenges/1}
\end{itemize}


\end{frame}

\end{document}\documentclass{beamer}
