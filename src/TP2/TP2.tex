%& --output-directory=../../pdf/

\documentclass{beamer}
\usepackage[frenchb]{babel}
\usepackage[T1]{fontenc}
\usepackage[utf8]{inputenc}
\usetheme{CambridgeUS}
\usepackage{tikz}
\usepackage{listings}
\usepackage{verbatim}
\usepackage{hyperref}
\usepackage{ulem}

\usetikzlibrary{arrows,shapes}
\lstset { %
	language=bash,
	backgroundcolor=\color{black!5}, % set backgroundcolor
	basicstyle=\footnotesize,% basic font setting
}

\title{IN104 - Solveur de Sudoku - TP2}
\author{Ugo Vollhardt}
\institute{CEA LIST}

\begin{document}
	\maketitle
	\section{Planning}
	\begin{frame}
	Déroulement prévisionnel : 
	\setbeamertemplate{enumerate items}[default]
	\begin{enumerate}
		\item Rappels sur Git, mise en place versionnement, structure donnée et premières fonctions de mises à jours de possibilités.
		\item Rappels/introduction à la notion de récursivité, poursuite du travail de la première séance et mise en place des premiers algo de résolution.
		\item Fin d'implémentation des permiers algo de résolution + rappels sur sujet au choix.
		\item Introduction à la librairie GTK+ pour Python et C, implémentation d'une interface minimale.
		\item Séance en autonomie.
		\item Revu des points précédents et approfondissement si besoin.
	\end{enumerate}
	\end{frame}
	\section{Gestion de fichiers}
	\begin{frame}
	\frametitle{Utilisation d'un fichier}
	\begin{columns}[c]
		\begin{column}{5cm}
			\begin{center}
				\begin{tikzpicture}[scale=1]
					\node[draw,rectangle,fill=gray!50] (Open) at (0,3) {Ouverture};
					\node[draw,rectangle,fill=gray!50] (interact) at (0,1.5) {Lecture/Écriture};
					\node[draw,rectangle,fill=gray!50] (close) at (0,0) {Fermeture};
					\draw[->,>=latex] (Open) to (interact);
					\draw[->,>=latex] (interact) to (close);
				\end{tikzpicture}
			\end{center}
		\end{column}
		\begin{column}{5cm}
			\setbeamertemplate{enumerate items}[default]
			\begin{enumerate}
				\item {\bfseries Ouverture} : Bloque l'accès au fichier. \\
				Crée un "canal" pour interagir directement avec son contenu.
				\item {\bfseries Lecture/Écriture} : A travers le "canal", modification ou non le contenu du fichier.
				\item {\bfseries Fermeture} : "Libère" le fichier. Si le fichier n'est pas fermé, il restera bloqué !
			\end{enumerate}
		\end{column}
	\end{columns}
	\end{frame}
	
	\begin{frame}%[fragile]
		\frametitle{Ouverture d'un fichier}
		\begin{itemize}
			\item  {\bfseries C} : \\
			 FILE* fopen(const char* nomDuFichier, const char* modeOuverture) 
			 \item  {\bfseries Python} : \\
			 file = open(name[, mode])
			 \item  {\bfseries MatLab} : \\
			 file = fopen(filename,permission)
		\end{itemize}
	\end{frame}

	\begin{frame}
	\frametitle{Mode d'ouverture}
		\begin{itemize}
			\item {\bfseries r} : Lecture seule, commence au début du fichier
			\item {\bfseries r+} : Lecture + Écriture, commence au début du fichier
			\item {\bfseries w} : Écriture seule, vide le fichier avant d'écrire ou crée un nouveau fichier
			\item {\bfseries w+} : Écriture + Lecture, vide le fichier avant d'écrire ou crée un nouveau fichier
			\item {\bfseries a} : Écriture seule, écrit à la suite du contenu du fichier ou crée un nouveau fichier
			\item {\bfseries a+} : Écriture + Lecture, écrit à la suite du contenu du fichier ou crée un nouveau fichier
		\end{itemize}
	\end{frame}

	\begin{frame}
	\frametitle{Écriture dans un fichier}
		\begin{itemize}
			\item  {\bfseries C} : \\
			int fputc(int caractere, FILE* pointeurSurFichier) : Écrit un seul caractère \\
			int fputs(const char* string, FILE* pointeurSurFichier) : Écrit une chaine de caractère \\
			int fprintf(FILE* pointeurSurFichier,...) : Même chose que printf
			\item  {\bfseries Python} : \\
			file.write(str) : Écrit une String dans le fichier\\
			file.writelines(sequence) : Écrit une sequence de String dans le fichier
			\item  {\bfseries MatLab} : \\
			fwrite(file,A) : Écrit le contenu de A
		\end{itemize}
	\end{frame}

	\begin{frame}
		\frametitle{Lecture d'un fichier}
		\begin{itemize}
			\item {\bfseries C} : \\
			int fgetc(FILE* pointeurDeFichier) : Lit un seul caractère \\
			char* fgets(char* chaine, int nbChar, FILE* pointeurSurFichier) : Lit nbChar et les place dans le tableau "chaine" \\
			fscanf(FILE* pointeurDeFichier,...) : Même chose que scanf 
			\item {\bfseries Python} : \\
			file.read([size]) : Lit "size" bits du fichier et les retourne sous la forme d'une string\\
			file.readline() : Lit une ligne complète et l'enregistre dans une string\\
			file.readlines() : Lit toutes les lignes et les enregistre dans une liste de string\\
			\item {\bfseries MatLab} : \\
			A = fread(fileID,sizeA) : Lit sizeA données et les place dans A \\
			A = fscanf(fileID,formatSpec) : Même chose que scanf en C
		\end{itemize}
	\end{frame}

	\begin{frame}%[fragile]
	\frametitle{fermeture d'un fichier}
	\begin{itemize}
		\item  {\bfseries C} : \\
		int fclose(FILE* pointeurSurFichier)
		\item  {\bfseries Python} : \\
		file.close()
		\item  {\bfseries MatLab} : \\
		fclose(file)
	\end{itemize}
	\end{frame}

	\section{Objectifs}
	\begin{frame}
	\frametitle{Objectifs de la séance}
	\setbeamertemplate{enumerate items}[default]
	\begin{enumerate}
		\item finir les objectifs de la séance précédente
		\item implémenter la boucle de résolution simple (mise à jours des possibilités constamment jusqu'à résolution de la grille ou impasse)
		\item Implémenter une fonction récursive pour les cas où la résolution tombe dans une impasse
	\end{enumerate}
\end{frame}

\end{document}