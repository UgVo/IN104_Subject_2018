%& --output-directory=../../pdf/

\documentclass{beamer}
\usepackage[frenchb]{babel}
\usepackage[T1]{fontenc}
\usepackage[utf8]{inputenc}
\usetheme{CambridgeUS}
\usepackage{tikz}
\usepackage{listings}
\usepackage{verbatim}
\usepackage{hyperref}
\usepackage{ulem}

\usetikzlibrary{arrows,shapes}
\lstset { %
	language=bash,
	backgroundcolor=\color{black!5}, % set backgroundcolor
	basicstyle=\footnotesize,% basic font setting
}

\title{IN104 - Solveur de Sudoku}
\author{Ugo Vollhardt}
\institute{CEA LIST}

\begin{document}
	\maketitle
	\section{Planning}
	\begin{frame}
	Déroulement prévisionnel : 
	\setbeamertemplate{enumerate items}[default]
	\begin{enumerate}
		\item Rappels sur Git, mise en place versionnement, structure donnée et premières fonctions de mises à jours de possibilités.
		\item Rappels/introduction à la notion de récursivité, poursuite du travail de la première séance et mise en place des premiers algo de résolution.
		\item Fin d'implémentation des permiers algo de résolution + rappels sur sujet au choix.
		\item Introduction à la librairie GTK+ pour Python et C, implémentation d'une interface minimale.
		\item Séance en autonomie.
		\item Revu des points précédents et approfondissement si besoin.
	\end{enumerate}
	\end{frame}
	\section{Projet Gitlab}
	\begin{frame}
	\frametitle{Pour Commencer : Mise en place du versionnement de code}
		\begin{itemize}
			\item Création d'un projet sur GitLab par l'un des deux membres du binôme.
			\item Ajout du fichier .gitignore correspondant au langage choisi
			\item Ajout du second membre du binôme en tant que master sur le projet GitLab
			\item Ajouter votre encadrant au projet avec les droits de développeur au moins (pseudo : UgoVollhardt)
			\item Vérifier que tous le monde a bien mis en place sa clef SSH
			\item Clonage du projet en local sur les ordinateurs de chacun
		\end{itemize}
	\end{frame}
	\section{Implémentation basique}
	\begin{frame}
	\frametitle{Grille de Sudoku : fonctionnement}
	\begin{columns}[c]
		\begin{column}{5cm}
			\begin{center}
				\begin{tikzpicture}[scale=0.5]
					\draw[step=1,black,thin,xshift=-0.5cm,yshift=-0.5cm] (0,0) grid (9,9);
					\draw[step=3,black,thick,xshift=-0.5cm,yshift=-0.5cm] (0,0) grid (9,9);
					\draw (2,0) node[]{7};
					\draw (2,1) node[]{5};
					\draw (0,1) node[]{2};
					\draw (0,2) node[]{6};
					\draw (1,2) node[]{1};
					\draw (3,0) node[]{6};
					\draw (4,1) node[]{7};
					\draw (4,2) node[]{4};
					\draw (3,2) node[]{8};
					\draw (3,0) node[]{6};
					\draw (6,0) node[]{8};
					\draw (7,0) node[]{9};
					\draw (7,1) node[]{6};
					\draw (8,1) node[]{1};
					\draw (8,4) node[]{2};
					\draw (7,4) node[]{5};
					\draw (6,5) node[]{6};
					\draw (5,5) node[]{8};
					\draw (4,5) node[]{5};
					\draw (4,4) node[]{3};
					\draw (4,3) node[]{9};
					\draw (3,3) node[]{2};
					\draw (2,3) node[]{3};
					\draw (1,4) node[]{4};
					\draw (0,4) node[]{8};
					\draw (0,7) node[]{1};
					\draw (1,7) node[]{5};
					\draw (1,8) node[]{9};
					\draw (2,8) node[]{2};
					\draw (5,8) node[]{4};
					\draw (4,7) node[]{6};
					\draw (4,6) node[]{1};
					\draw (5,6) node[]{2};
					\draw (6,8) node[]{7};
					\draw (6,7) node[]{2};
					\draw (7,6) node[]{4};
					\draw (8,6) node[]{9};
					\draw (8,7) node[]{8};
				\end{tikzpicture}
			\end{center}
		\end{column}
		\begin{column}{5cm}
		\end{column}
	\end{columns}
	\end{frame}

\begin{frame}
\frametitle{Grille de Sudoku : fonctionnement}
\begin{columns}[c]
	\begin{column}{5cm}
		\begin{center}
			\begin{tikzpicture}[scale=0.5]
			\fill[red!50,thick] (4.5,4.5) rectangle (5.5,3.5);
			\draw[step=1,black,thin,xshift=-0.5cm,yshift=-0.5cm] (0,0) grid (9,9);
			\draw[step=3,black,thick,xshift=-0.5cm,yshift=-0.5cm] (0,0) grid (9,9);
			\draw (2,0) node[]{7};
			\draw (2,1) node[]{5};
			\draw (0,1) node[]{2};
			\draw (0,2) node[]{6};
			\draw (1,2) node[]{1};
			\draw (3,0) node[]{6};
			\draw (4,1) node[]{7};
			\draw (4,2) node[]{4};
			\draw (3,2) node[]{8};
			\draw (3,0) node[]{6};
			\draw (6,0) node[]{8};
			\draw (7,0) node[]{9};
			\draw (7,1) node[]{6};
			\draw (8,1) node[]{1};
			\draw (8,4) node[]{2};
			\draw (7,4) node[]{5};
			\draw (6,5) node[]{6};
			\draw (5,5) node[]{8};
			\draw (4,5) node[]{5};
			\draw (4,4) node[]{3};
			\draw (4,3) node[]{9};
			\draw (3,3) node[]{2};
			\draw (2,3) node[]{3};
			\draw (1,4) node[]{4};
			\draw (0,4) node[]{8};
			\draw (0,7) node[]{1};
			\draw (1,7) node[]{5};
			\draw (1,8) node[]{9};
			\draw (2,8) node[]{2};
			\draw (5,8) node[]{4};
			\draw (4,7) node[]{6};
			\draw (4,6) node[]{1};
			\draw (5,6) node[]{2};
			\draw (6,8) node[]{7};
			\draw (6,7) node[]{2};
			\draw (7,6) node[]{4};
			\draw (8,6) node[]{9};
			\draw (8,7) node[]{8};
			
			\end{tikzpicture}
		\end{center}
	\end{column}
	\begin{column}{5cm}
	\end{column}
\end{columns}
\end{frame}

	\begin{frame}
	\frametitle{Grille de Sudoku : fonctionnement}
	\begin{columns}[c]
		\begin{column}{5cm}
			\begin{center}
				\begin{tikzpicture}[scale=0.5]
				\draw[step=1,black,thin,xshift=-0.5cm,yshift=-0.5cm] (0,0) grid (9,9);
				\draw[step=3,black,thick,xshift=-0.5cm,yshift=-0.5cm] (0,0) grid (9,9);
				\draw (2,0) node[]{7};
				\draw (2,1) node[]{5};
				\draw (0,1) node[]{2};
				\draw (0,2) node[]{6};
				\draw (1,2) node[]{1};
				\draw (3,0) node[]{6};
				\draw (4,1) node[]{7};
				\draw (4,2) node[]{4};
				\draw (3,2) node[]{8};
				\draw (3,0) node[]{6};
				\draw (6,0) node[]{8};
				\draw (7,0) node[]{9};
				\draw (7,1) node[]{6};
				\draw (8,1) node[]{1};
				\draw (8,4) node[]{2};
				\draw (7,4) node[]{5};
				\draw (6,5) node[]{6};
				\draw (5,5) node[]{8};
				\draw (4,5) node[]{5};
				\draw (4,4) node[]{3};
				\draw (4,3) node[]{9};
				\draw (3,3) node[]{2};
				\draw (2,3) node[]{3};
				\draw (1,4) node[]{4};
				\draw (0,4) node[]{8};
				\draw (0,7) node[]{1};
				\draw (1,7) node[]{5};
				\draw (1,8) node[]{9};
				\draw (2,8) node[]{2};
				\draw (5,8) node[]{4};
				\draw (4,7) node[]{6};
				\draw (4,6) node[]{1};
				\draw (5,6) node[]{2};
				\draw (6,8) node[]{7};
				\draw (6,7) node[]{2};
				\draw (7,6) node[]{4};
				\draw (8,6) node[]{9};
				\draw (8,7) node[]{8};
				\fill[red!50,thick] (4.5,4.5) rectangle (5.5,3.5);
				\draw[color=blue!50,very thick] (-0.5,4.5) rectangle (8.5,3.5);
				%\draw[color=cyan!50,very thick] (4.5,8.5) rectangle (5.5,-0.5);
				%\draw[color=orange!50,very thick] (2.5,5.5) rectangle (5.5,2.5);
				
				\end{tikzpicture}
			\end{center}
		\end{column}
		\begin{column}{5cm}
			\begin{tikzpicture}[scale=0.5]
			%\draw[step=1,black,thin,xshift=-0.5cm,yshift=-0.5cm] (0,-1) grid (9,0);
			%\foreach \x in {1,6,7} {\draw (\x-1,-1) node[]{\x};}
			
			\draw[step=1,color=blue!50,thick,xshift=-0.5cm,yshift=-0.5cm] (0,2) grid (9,3);
			\foreach \x in {1,6,7,9} {\draw (\x-1,2) node[]{\x};}
			\foreach \x in {2,5,3,4,8} {\draw (\x-1,2) node[]{\xout{\x}};}
			
			%\draw[step=1,color=cyan!50,thick,xshift=-0.5cm,yshift=-0.5cm] (0,4) grid (9,5);
			%\foreach \x in {1,3,5,6,7,9} {\draw (\x-1,4) node[]{\x};}
			%\foreach \x in {2,4,8} {\draw (\x-1,4) node[]{\xout{\x}};}
			
			%\draw[step=1,color=orange!50,thick,xshift=-0.5cm,yshift=-0.5cm] (0,6) grid (9,7);
			%\foreach \x in {1,4,6,7} {\draw (\x-1,6) node[]{\x};}
			%\foreach \x in {2,3,5,8,9} {\draw (\x-1,6) node[]{\xout{\x}};}
			\end{tikzpicture}
		\end{column}
	\end{columns}
	\end{frame}

	\begin{frame}
	\frametitle{Grille de Sudoku : fonctionnement}
	\begin{columns}[c]
		\begin{column}{5cm}
			\begin{center}
				\begin{tikzpicture}[scale=0.5]
				\draw[step=1,black,thin,xshift=-0.5cm,yshift=-0.5cm] (0,0) grid (9,9);
				\draw[step=3,black,thick,xshift=-0.5cm,yshift=-0.5cm] (0,0) grid (9,9);
				\draw (2,0) node[]{7};
				\draw (2,1) node[]{5};
				\draw (0,1) node[]{2};
				\draw (0,2) node[]{6};
				\draw (1,2) node[]{1};
				\draw (3,0) node[]{6};
				\draw (4,1) node[]{7};
				\draw (4,2) node[]{4};
				\draw (3,2) node[]{8};
				\draw (3,0) node[]{6};
				\draw (6,0) node[]{8};
				\draw (7,0) node[]{9};
				\draw (7,1) node[]{6};
				\draw (8,1) node[]{1};
				\draw (8,4) node[]{2};
				\draw (7,4) node[]{5};
				\draw (6,5) node[]{6};
				\draw (5,5) node[]{8};
				\draw (4,5) node[]{5};
				\draw (4,4) node[]{3};
				\draw (4,3) node[]{9};
				\draw (3,3) node[]{2};
				\draw (2,3) node[]{3};
				\draw (1,4) node[]{4};
				\draw (0,4) node[]{8};
				\draw (0,7) node[]{1};
				\draw (1,7) node[]{5};
				\draw (1,8) node[]{9};
				\draw (2,8) node[]{2};
				\draw (5,8) node[]{4};
				\draw (4,7) node[]{6};
				\draw (4,6) node[]{1};
				\draw (5,6) node[]{2};
				\draw (6,8) node[]{7};
				\draw (6,7) node[]{2};
				\draw (7,6) node[]{4};
				\draw (8,6) node[]{9};
				\draw (8,7) node[]{8};
				\fill[red!50,thick] (4.5,4.5) rectangle (5.5,3.5);
				\draw[color=blue!50,very thick] (-0.5,4.5) rectangle (8.5,3.5);
				\draw[color=cyan!50,very thick] (4.5,8.5) rectangle (5.5,-0.5);
				%\draw[color=orange!50,very thick] (2.5,5.5) rectangle (5.5,2.5);
				
				\end{tikzpicture}
			\end{center}
		\end{column}
		\begin{column}{5cm}
			\begin{tikzpicture}[scale=0.5]
			%\draw[step=1,black,thin,xshift=-0.5cm,yshift=-0.5cm] (0,-1) grid (9,0);
			%\foreach \x in {1,6,7} {\draw (\x-1,-1) node[]{\x};}
			
			\draw[step=1,color=blue!50,thick,xshift=-0.5cm,yshift=-0.5cm] (0,2) grid (9,3);
			\foreach \x in {1,6,7,9} {\draw (\x-1,2) node[]{\x};}
			\foreach \x in {2,5,3,4,8} {\draw (\x-1,2) node[]{\xout{\x}};}
			
			\draw[step=1,color=cyan!50,thick,xshift=-0.5cm,yshift=-0.5cm] (0,4) grid (9,5);
			\foreach \x in {1,3,5,6,7,9} {\draw (\x-1,4) node[]{\x};}
			\foreach \x in {2,4,8} {\draw (\x-1,4) node[]{\xout{\x}};}
			
			%\draw[step=1,color=orange!50,thick,xshift=-0.5cm,yshift=-0.5cm] (0,6) grid (9,7);
			%\foreach \x in {1,4,6,7} {\draw (\x-1,6) node[]{\x};}
			%\foreach \x in {2,3,5,8,9} {\draw (\x-1,6) node[]{\xout{\x}};}
			\end{tikzpicture}
		\end{column}
	\end{columns}
	\end{frame}

	\begin{frame}
	\frametitle{Grille de Sudoku : fonctionnement}
	\begin{columns}[c]
		\begin{column}{5cm}
			\begin{center}
				\begin{tikzpicture}[scale=0.5]
				\draw[step=1,black,thin,xshift=-0.5cm,yshift=-0.5cm] (0,0) grid (9,9);
				\draw[step=3,black,thick,xshift=-0.5cm,yshift=-0.5cm] (0,0) grid (9,9);
				\draw (2,0) node[]{7};
				\draw (2,1) node[]{5};
				\draw (0,1) node[]{2};
				\draw (0,2) node[]{6};
				\draw (1,2) node[]{1};
				\draw (3,0) node[]{6};
				\draw (4,1) node[]{7};
				\draw (4,2) node[]{4};
				\draw (3,2) node[]{8};
				\draw (3,0) node[]{6};
				\draw (6,0) node[]{8};
				\draw (7,0) node[]{9};
				\draw (7,1) node[]{6};
				\draw (8,1) node[]{1};
				\draw (8,4) node[]{2};
				\draw (7,4) node[]{5};
				\draw (6,5) node[]{6};
				\draw (5,5) node[]{8};
				\draw (4,5) node[]{5};
				\draw (4,4) node[]{3};
				\draw (4,3) node[]{9};
				\draw (3,3) node[]{2};
				\draw (2,3) node[]{3};
				\draw (1,4) node[]{4};
				\draw (0,4) node[]{8};
				\draw (0,7) node[]{1};
				\draw (1,7) node[]{5};
				\draw (1,8) node[]{9};
				\draw (2,8) node[]{2};
				\draw (5,8) node[]{4};
				\draw (4,7) node[]{6};
				\draw (4,6) node[]{1};
				\draw (5,6) node[]{2};
				\draw (6,8) node[]{7};
				\draw (6,7) node[]{2};
				\draw (7,6) node[]{4};
				\draw (8,6) node[]{9};
				\draw (8,7) node[]{8};
				\fill[red!50,thick] (4.5,4.5) rectangle (5.5,3.5);
				\draw[color=blue!50,very thick] (-0.5,4.5) rectangle (8.5,3.5);
				\draw[color=cyan!50,very thick] (4.5,8.5) rectangle (5.5,-0.5);
				\draw[color=orange!50,very thick] (2.5,5.5) rectangle (5.5,2.5);
				
				\end{tikzpicture}
			\end{center}
		\end{column}
		\begin{column}{5cm}
			\begin{tikzpicture}[scale=0.5]
			%\draw[step=1,black,thin,xshift=-0.5cm,yshift=-0.5cm] (0,-1) grid (9,0);
			%\foreach \x in {1,6,7} {\draw (\x-1,-1) node[]{\x};}
			
			\draw[step=1,color=blue!50,thick,xshift=-0.5cm,yshift=-0.5cm] (0,2) grid (9,3);
			\foreach \x in {1,6,7,9} {\draw (\x-1,2) node[]{\x};}
			\foreach \x in {2,5,3,4,8} {\draw (\x-1,2) node[]{\xout{\x}};}
			
			\draw[step=1,color=cyan!50,thick,xshift=-0.5cm,yshift=-0.5cm] (0,4) grid (9,5);
			\foreach \x in {1,3,5,6,7,9} {\draw (\x-1,4) node[]{\x};}
			\foreach \x in {2,4,8} {\draw (\x-1,4) node[]{\xout{\x}};}
			
			\draw[step=1,color=orange!50,thick,xshift=-0.5cm,yshift=-0.5cm] (0,6) grid (9,7);
			\foreach \x in {1,4,6,7} {\draw (\x-1,6) node[]{\x};}
			\foreach \x in {2,3,5,8,9} {\draw (\x-1,6) node[]{\xout{\x}};}
			\end{tikzpicture}
		\end{column}
	\end{columns}
	\end{frame}

	\begin{frame}
	\frametitle{Grille de Sudoku : fonctionnement}
		\begin{columns}[c]
			\begin{column}{5cm}
				\begin{center}
					\begin{tikzpicture}[scale=0.5]
					\draw[step=1,black,thin,xshift=-0.5cm,yshift=-0.5cm] (0,0) grid (9,9);
					\draw[step=3,black,thick,xshift=-0.5cm,yshift=-0.5cm] (0,0) grid (9,9);
					\draw (2,0) node[]{7};
					\draw (2,1) node[]{5};
					\draw (0,1) node[]{2};
					\draw (0,2) node[]{6};
					\draw (1,2) node[]{1};
					\draw (3,0) node[]{6};
					\draw (4,1) node[]{7};
					\draw (4,2) node[]{4};
					\draw (3,2) node[]{8};
					\draw (3,0) node[]{6};
					\draw (6,0) node[]{8};
					\draw (7,0) node[]{9};
					\draw (7,1) node[]{6};
					\draw (8,1) node[]{1};
					\draw (8,4) node[]{2};
					\draw (7,4) node[]{5};
					\draw (6,5) node[]{6};
					\draw (5,5) node[]{8};
					\draw (4,5) node[]{5};
					\draw (4,4) node[]{3};
					\draw (4,3) node[]{9};
					\draw (3,3) node[]{2};
					\draw (2,3) node[]{3};
					\draw (1,4) node[]{4};
					\draw (0,4) node[]{8};
					\draw (0,7) node[]{1};
					\draw (1,7) node[]{5};
					\draw (1,8) node[]{9};
					\draw (2,8) node[]{2};
					\draw (5,8) node[]{4};
					\draw (4,7) node[]{6};
					\draw (4,6) node[]{1};
					\draw (5,6) node[]{2};
					\draw (6,8) node[]{7};
					\draw (6,7) node[]{2};
					\draw (7,6) node[]{4};
					\draw (8,6) node[]{9};
					\draw (8,7) node[]{8};
					\fill[red!50,thick] (4.5,4.5) rectangle (5.5,3.5);
					\draw[color=blue!50,very thick] (-0.5,4.5) rectangle (8.5,3.5);
					\draw[color=cyan!50,very thick] (4.5,8.5) rectangle (5.5,-0.5);
					\draw[color=orange!50,very thick] (2.5,5.5) rectangle (5.5,2.5);
					
					\end{tikzpicture}
				\end{center}
			\end{column}
			\begin{column}{5cm}
				\begin{tikzpicture}[scale=0.5]
				\draw[step=1,black,thin,xshift=-0.5cm,yshift=-0.5cm] (0,-1) grid (9,0);
				\foreach \x in {1,6,7} {\draw (\x-1,-1) node[]{\x};}
				
				\draw[step=1,color=blue!50,thick,xshift=-0.5cm,yshift=-0.5cm] (0,2) grid (9,3);
				\foreach \x in {1,6,7,9} {\draw (\x-1,2) node[]{\x};}
				\foreach \x in {2,5,3,4,8} {\draw (\x-1,2) node[]{\xout{\x}};}
				
				\draw[step=1,color=cyan!50,thick,xshift=-0.5cm,yshift=-0.5cm] (0,4) grid (9,5);
				\foreach \x in {1,3,5,6,7,9} {\draw (\x-1,4) node[]{\x};}
				\foreach \x in {2,4,8} {\draw (\x-1,4) node[]{\xout{\x}};}
				
				\draw[step=1,color=orange!50,thick,xshift=-0.5cm,yshift=-0.5cm] (0,6) grid (9,7);
				\foreach \x in {1,4,6,7} {\draw (\x-1,6) node[]{\x};}
				\foreach \x in {2,3,5,8,9} {\draw (\x-1,6) node[]{\xout{\x}};}
				\end{tikzpicture}
			\end{column}
		\end{columns}
	\end{frame}
	\section{Objectifs}
	\begin{frame}
	\frametitle{Objectifs de la séance}
	\setbeamertemplate{enumerate items}[default]
	\begin{enumerate}
		\item Choisir et implémenter les structures de données (représentation de la grille de Sudoku et des possibilités pour chaque cases)
		\item Avoir un affichage minimal dans le terminal d'une grille
		\item Coder les fonctions pour mettre à jour les possibilités pour chaque cases.
		\begin{itemize}
			\item Possibilités sur la colonne
			\item Possibilités sur la ligne
			\item Possibilités dans le carré
			\item Intersection de ces trois ensembles
		\end{itemize}
	\end{enumerate}
\end{frame}

\end{document}